\documentclass{article}
\usepackage[utf8]{inputenc}
\usepackage{hyperref}
\usepackage{float}
\usepackage{graphicx} % Gestione immagini
\usepackage{amsmath} %simboli matematici
\usepackage{minted}
\usepackage{xcolor} % to access the named colour LightGray
\definecolor{LightGray}{gray}{0.9}

\title{Esercizi 23-11-06}
\author{Riccardo Gennaro}
\date{November 2023}

\begin{document}

\maketitle

\section*{Esercizio 1}

    Scrivere un programma che gestisca l'anagrafica di un set di utenti.

    In particolare, il programma deve poter:
    \begin{itemize}
        \item aggiungere un l'anagrafica di un utente;
        \item stampare le anagrafiche ordinate per nome;
        \item stampare le anagrafiche ordinate per cognome;
        \item cercare un anagrafica per nome (corrispondenza esatta);
        \item cercare un anagrafica per cognome (corrispondenza esatta);
    \end{itemize}

    Con anagrafica si intende:

    \begin{itemize}
        \item nome;
        \item cognome;
        \item indirizzo, composto da:
        \begin{itemize}
            \item via;
            \item civico;
            \item comune;
            \item CAP;
            \item provincia;
        \end{itemize}
    \end{itemize}

    \noindent Descrivere l'anagrafica tramite \textit{struct}.    

    \noindent Il programma prevede delle dimensioni massime per gli array.

    L'utente deve poter scegliere una delle opzioni sopra. Il programma termina quando l'utente inserisce la stringa '\texttt{exit}' nel menù.
    Scrivere il programma rispettando i principi della programmazione su file multipli. 

    \newpage

\section*{Esercizio 2}

Scrivere un programma che data una matrice bidimensionale con dimensioni inserite dall'utente, la ruoti di 90° clockwise. Per esempio,
    \[
    \begin{bmatrix}
        11 & 22 & 33\\
        44 & 55 & 66\\
        77 & 88 & 99
    \end{bmatrix} \]
    
    \noindent verrà ruotata nel modo seguente
    
    \[
    \begin{bmatrix}
        77 & 44 & 11\\
        88 & 55 & 22\\
        99 & 66 & 33
    \end{bmatrix} \]

Il programma presenta matrice e dimensione hardcoded.

\end{document}