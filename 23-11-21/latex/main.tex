\documentclass{article}
\usepackage[utf8]{inputenc}
\usepackage{hyperref}
\usepackage{float}
\usepackage{graphicx} % Gestione immagini
\usepackage{amsmath} %simboli matematici
\usepackage{minted}
\usepackage{xcolor} % to access the named colour LightGray

\title{Esercizi 13-11-23}
\author{Riccardo Gennaro}
\date{November 2023}

\begin{document}

    \maketitle

    \section*{NOTA}

    Tutti gli esercizi vanno svolti rispettando le buone pratiche di allocazione dinamica.
    Potete implementare su più file per esercitarvi.

    \section*{Esercizio 1 (Coda a priorità)}

    Un messaggio viene rappresentato mediante una struct

    \begin{minted}
    [
        frame=lines,
        framesep=2mm,
        baselinestretch=1.2,
        fontsize=\footnotesize,
        linenos
    ]
    {C}
        struct messaggio {
           char testo[10000];
           int priorita;
        }
    \end{minted}

    dove priorità ha un range crescente da 1 a 10.
    Scrivere una struttura dati \("\)coda a priorita\("\) in cui e' possibile inserire messaggi, da cui vengono estratti in modo FIFO per classi di priorita': (prima quelli a priorita' 10, poi quelli a priorita' 9,...)

    \section*{Esercizio 1 bis (Anagrafica)}

    Scrivere un programma che gestisca l'anagrafica di un set di utenti.

    In particolare, il programma deve poter:
    \begin{itemize}
        \item aggiungere un l'anagrafica di un utente;
        \item stampare le anagrafiche ordinate per nome;
        \item stampare le anagrafiche ordinate per cognome;
        \item cercare un anagrafica per nome (corrispondenza esatta);
        \item cercare un anagrafica per cognome (corrispondenza esatta);
    \end{itemize}

    Con anagrafica si intende:

    \begin{itemize}
        \item nome;
        \item cognome;
        \item indirizzo, composto da:
        \begin{itemize}
            \item via;
            \item civico;
            \item comune;
            \item CAP;
            \item provincia;
        \end{itemize}
    \end{itemize}

    \noindent Descrivere l'anagrafica tramite \textit{struct}.

    \noindent Il programma prevede delle dimensioni massime per gli array.

    L'utente deve poter scegliere una delle opzioni sopra.
    Il programma termina quando l'utente inserisce la stringa \('\)\texttt{exit}\('\) nel menù.
    Scrivere il programma rispettando i principi della programmazione su file multipli.

    \section*{Esercizio 2 (Correzione testo)}

    Scrivere un programma, nel file esercizio1.cc, che, presi come argomenti del main i nomi di due file, copi il primo file nel secondo correggendone la sintassi e generando in tal modo un testo “corretto” secondo le seguenti regole:

    \begin{itemize}
        \item la prima parola del testo deve iniziare con una lettera maiuscola;
        \item tutte le parole che seguono i seguenti caratteri: “.”, “?” e “!”, devono iniziare con una lettera maiuscola.
    \end{itemize}

    Se ad esempio l'eseguibile è a.out, il comando \texttt{./a.out testo testocorretto} creerà un nuovo file di nome \texttt{testocorretto} e vi copierà il contenuto del file dato testo, modificando le parole quando queste non verificano le regole descritte sopra. Nelle figure 1 e 2 un esempio di file \texttt{testo} e \texttt{testocorretto}.\\

    Esempio di input:
    \begin{verbatim}
filastrocca delle parole:
Fatevi avanti! chi ne vuole?
di parole ho la testa piena,
con dentro la 'luna' e la 'balena'.
ci sono parole per gli amici:
Buon giorno, Buon anno, Siate felici!
parole belle e parole buone;
parole per ogni sorta di persone.
di G. Rodari.
    \end{verbatim}

    Relativo output:
    \begin{verbatim}
Filastrocca delle parole:
Fatevi avanti! Chi ne vuole?
Di parole ho la testa piena,
con dentro la 'luna' e la 'balena'.
Ci sono parole per gli amici:
Buon giorno, Buon anno, Siate felici!
Parole belle e parole buone;
parole per ogni sorta di persone.
Di G. Rodari.
    \end{verbatim}

    NOTA 1: Per semplicità si assuma che il testo contenuto nel primo file inizi con un carattere alfabetico, non contenga “\ldots” e che “.”, “?” e “!” siano sempre preceduti da una parola e seguiti da uno spazio.

    NOTA 2: Per semplicità si assuma che ogni parola contenuta nel testo del primo file abbia al massimo lunghezza 30 caratteri.

    NOTA 3: E' ammesso l'uso della funzione strlen della libreria \texttt{cstring}, non è ammesso l'uso di altre funzioni di libreria, in particolare della funzione toupper.

    NOTA 4: il programma deve potenzialmente funzionare con ogni possibile codifica dei caratteri secondo le regole di tali codifiche viste a lezione (quindi non solo ASCII). Per realizzare la conversione da caratteri minuscoli in maiuscoli, è vietato l'uso di tabelle o di 26 if o switch-case, uno per ogni carattere.

    \section*{Esercizio 3 (Matrici)}

    Scrivere un programma che calcoli il \textit{prodotto matriciale} di due matrici di numeri interi.
    Se le matrici non sono compatibili per il prodotto stampare un messaggio di errore.\\
    \noindent La dimensione delle matrici deve essere specificata dall'utente.
    Il prodotto è calcolato con la funzione seguente:
    \begin{minted}
    [
        frame=lines,
        framesep=2mm,
        baselinestretch=1.2,
        fontsize=\footnotesize,
        linenos
    ]
    {C}
    int** compute\textunderscore prod(int** m1, int r1, int c1, int** m2, int c2);
    \end{minted}

    Dove
    \begin{itemize}
        \item \texttt{m1} è la prima matrice;
        \item \texttt{r1} è il numero di righe di m1;
        \item \texttt{c1} è il numero di colonne di m1;
        \item \texttt{m2} è la seconda matrice;
        \item \texttt{c2} è il numero di colonne di m2;
    \end{itemize}
    NOTA: la complessità di \texttt{compute\textunderscore prod(...)} deve essere pari o inferiore a $O(n^3)$.
    Ciò significa che potete implementare la funzione con l'algoritmo che si basa sulla definizione del prodotto fra matrici, oppure, se siete dei pazzi maniaci, potete implementare l'\color{blue}\href{https://it.wikipedia.org/wiki/Algoritmo\textunderscore di\textunderscore Strassen}{algoritmo di Strassen}\color{black}.

\end{document}
