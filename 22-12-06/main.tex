\documentclass{article}
\usepackage[utf8]{inputenc}
\usepackage{hyperref}
\usepackage{float}
\usepackage{graphicx} % Gestione immagini
\usepackage{amsmath} %simboli matematici

\title{Esercizi 07-12-22}
\author{Riccardo Gennaro}
\date{November 2022}

\begin{document}

\maketitle

\section*{NOTA}

Tutti gli esercizi vanno svolti rispettando le buone pratiche di allocazione dinamica. Potete implementare su più file per esercitarvi.

\section*{Esercizio 1}

Un messaggio viene rappresentato mediante una struct \\

\begin{verbatim}
struct messaggio {
   char testo[10000];
   int priorita;
}    
\end{verbatim}

dove priorità ha un range crescente da 1 a 10.
Scrivere una struttura dati "coda a priorita" in cui e' possibile inserire messaggi, da cui vengono estratti in modo FIFO per classi di priorita': (prima quelli a priorita' 10, poi quelli a priorita' 9,...)

\section*{Esercizio 2}

Scrivere un programma, nel file esercizio1.cc, che, presi come argomenti del main i nomi di due file, copi il primo file nel secondo correggendone la sintassi e generando in tal modo un testo “corretto” secondo le seguenti regole:

\begin{itemize}
   \item la prima parola del testo deve iniziare con una lettera maiuscola;
   \item tutte le parole che seguono i seguenti caratteri: “.”, “?” e “!”, devono iniziare con una lettera maiuscola.
\end{itemize}

Se ad esempio l'eseguibile è a.out, il comando \texttt{./a.out testo testocorretto} creerà un nuovo file di nome \texttt{testocorretto} e vi copierà il contenuto del file dato testo, modificando le parole quando queste non verificano le regole descritte sopra. Nelle figure 1 e 2 un esempio di file \texttt{testo} e \texttt{testocorretto}.

\begin{table}[h!]
   \begin{minipage}[c]{0.45\linewidth}
      \begin{verbatim}
filastrocca delle parole:
Fatevi avanti! chi ne vuole?
di parole ho la testa piena,
con dentro la “luna” e la “balena”.
ci sono parole per gli amici:
Buon giorno, Buon anno, Siate felici!
parole belle e parole buone;
parole per ogni sorta di persone.
di G. Rodari.
      \end{verbatim}
      \captionsetup{justification=centering}
      \captionof{figure}{testo}
   \end{minipage}
   \hspace{0.5cm}
   \begin{minipage}[c]{0.45\linewidth}
      \begin{verbatim}
filastrocca delle parole:
Fatevi avanti! chi ne vuole?
di parole ho la testa piena,
con dentro la “luna” e la “balena”.
ci sono parole per gli amici:
Buon giorno, Buon anno, Siate felici!
parole belle e parole buone;
parole per ogni sorta di persone.
di G. Rodari.
         \end{verbatim}
      \captionsetup{justification=centering}
      \caption{testocorretto}
   \end{minipage}
\end{table}

NOTA 1: Per semplicità si assuma che il testo contenuto nel primo file inizi con un carattere alfabetico, non contenga “...” e che “.”, “?” e “!” siano sempre preceduti da una parola e seguiti da uno spazio.

NOTA 2: Per semplicità si assuma che ogni parola contenuta nel testo del primo file abbia al massimo lunghezza 30 caratteri.

NOTA 3: E' ammesso l'uso della funzione strlen della libreria <cstring>, non è ammesso l'uso di altre funzioni di libreria, in particolare della funzione toupper.

NOTA 4: il programma deve potenzialmente funzionare con ogni possibile codifica dei caratteri secondo le regole di tali codifiche viste a lezione (quindi non solo ASCII). Per realizzare la conversione da caratteri minuscoli in maiuscoli, è vietato l'uso di tabelle o di 26 if o switch-case, uno per ogni carattere.

\end{document}