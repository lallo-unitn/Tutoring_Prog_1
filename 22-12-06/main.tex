\documentclass{article}
\usepackage[utf8]{inputenc}
\usepackage{hyperref}
\usepackage{float}
\usepackage{graphicx} % Gestione immagini
\usepackage{amsmath} %simboli matematici
\usepackage{minted}
\usepackage{xcolor} % to access the named colour LightGray
\definecolor{LightGray}{gray}{0.9}

\title{Esercizi 22-11-11}
\author{Riccardo Gennaro & Jan Tomassi}
\date{November 2022}

\begin{document}

\maketitle

\section*{NOTA}

Tutti gli esercizi vanno svolti rispettando le buone pratiche di allocazione dinamica. Potete implementare su più file per esercitarvi.

\section*{Esercizio 1}

Un messaggio viene rappresentato mediante una struct \\

\begin{verbatim}
struct messaggio {
   char testo[10000];
   int priorita;
}    
\end{verbatim}

dove priorità ha un range crescente da 1 a 10.
Scrivere una struttura dati "coda a priorita" in cui e' possibile inserire messaggi, da cui vengono estratti in modo FIFO per classi di priorita': (prima quelli a priorita' 10, poi quelli a priorita' 9,...)

\section*{Esercizio 2}

    

\end{document}