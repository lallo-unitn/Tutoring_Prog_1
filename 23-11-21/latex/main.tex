\documentclass{article}
\usepackage[utf8]{inputenc}
\usepackage{hyperref}
\usepackage{float}
\usepackage{graphicx} % Gestione immagini
\usepackage{amsmath} %simboli matematici
\usepackage{minted}
\usepackage{xcolor} % to access the named colour LightGray

\title{Esercizi 21-11-23}
\author{Riccardo Gennaro}
\date{November 2023}

\begin{document}

    \maketitle

    \section*{Esercizio 1 (Archivio studenti)}

    
    Sviluppa un programma per gestire un archivio di studenti utilizzando una struttura \texttt{Student} che descrive nome, cognome e id, e una struttura \texttt{Archivio} che contiene gli studenti ordinati per cognome e id. Per mantenere due ordinamenti, la struttura \texttt{Archivio} usa due array.
    Si richiedono funzioni per l'aggiunta e la stampa degli studenti.
    Non è richiesto l'input dall'utente, potete usare il file \texttt{01-esercizio\textunderscore main.c|.cpp} per il testing.

\end{document}
