\documentclass{article}
\usepackage[utf8]{inputenc}
\usepackage{hyperref}
\usepackage{float}
\usepackage{graphicx} % Gestione immagini
\usepackage{amsmath} %simboli matematici
\usepackage{minted}
\usepackage{xcolor} % to access the named colour LightGray
\definecolor{LightGray}{gray}{0.9}

\title{Esercizi 23-10-30}
\author{Riccardo Gennaro}
\date{October 2023}

\begin{document}

\maketitle

\section*{Esercizio 1}

    Scrivere la funzione

    \begin{minted}
    [
        frame=lines,
        framesep=2mm,
        baselinestretch=1.2,
        fontsize=\footnotesize,
        linenos
    ]
    {C}
    void outerProduct3(float p[][3], float a[], float b[]);
    \end{minted}
    
    che ritorna il \textit{prodotto esterno} dei primi tre elementi di \textbf{a} e \textbf{b}. Tale \textit{prodotto esterno} è definito come la matrice p[3][3] i cui elementi $p_(ij)$ sono tali che 
        \[p_{ij} = a_i \cdot b_j\]

\section*{Esercizio 2}

    Scrivere un programma che, dato un file di input
    contente delle parole, generi un secondo file in output che contenga le stesse
    parole, ma in ordine inverso rispetto al file iniziale. Per essere copiata, una
    parola deve essere composta da un numero di caratteri \textbf{pari}.\\
    
    \noindent Il programma dovrà accettare due parametri da riga di comando: il nome
    del file in input e il nome del file su cui effettuare l'output, in questo ordine.\\
    
    \noindent Il programma dovrà anche implementare dei controlli sul numero di argomenti
    passati da riga di comando e sull'apertura dei file (in caso di un file
    di input non esistente).\\
    
    \noindent Procediamo ora con un esempio. Dato in input il file input\textunderscore A, contenente le seguenti parole:\\
    
    \noindent \texttt{Impara a risolvere tutti i problemi che sono stati risolti.
    Richard Feynman}\\
    \newpage
    \noindent se l'eseguibile è a.out, allora il comando \texttt{./a.out input\textunderscore A output} genererà un file chiamato \texttt{output} che conterrà le seguenti parole:\\
    
    \noindent \texttt{risolti. sono problemi Impara}\\
    
    \noindent \textbf{Note}
    \begin{itemize}
        \item Per semplicità, considerate come una \textit{parola} qualsiasi stringa di caratteri
        compresa tra due spazi bianchi (e/o separatori di tabulazione,
        nuova linea e fine file). Quindi, sono da considerarsi parole anche
        stringhe come \texttt{"andato:"} o \texttt{"!esempio"}.
        \item Il file in input può contenere al massimo 10000 parole e ognuna di
        queste parole è formata da al massimo 100 caratteri.
        \item Non è consentito l'utilizzo di librerie diverse da \texttt{<iostream>} ed \texttt{<fstream>} pena \textbf{l'annullamento dell'esercizio}.
        \item E' consentito definire ed implementare funzioni ausiliarie che possano\\
        aiutarvi nella soluzione del problema.
    \end{itemize}

\section*{Esercizio 3}

    Scrivere nel file esercizio3.cc:\\
    
    \noindent \begin{itemize}
        \item una funzione \textbf{ricorsiva} \textit{calcola\textunderscore norma\textunderscore ricorsivo} che, dato un array di long e la sua dimensione, restituisca un numero \textit{double} corrispondente alla \textbf{norma ad uno} dell’array stesso;
        \item una funzione \textbf{ricorsiva} \textit{normalizza} che, dato un array di \textit{long} e la sua dimensione, restituisca un \textbf{nuovo array} contenente, per ogni elemento, l’elemento corrispondente del primo array diviso per la \textbf{norma ad uno} dell’array in input, calcolata con la funzione di cui al punto precedente.
        
    \end{itemize}
        
    \noindent Le operazioni di calcolo della norma e di divisione degli elementi dell’array per il valore della norma vanno eseguite tramite funzioni ricorsive, \textbf{pena l’annullamento dell’esercizio}.\\
    
    \noindent \textbf{Non è ammesso l’uso di cicli, variabili globali o static.} È ammesso l’uso di funzioni ausiliarie, purché ricorsive. È consentito l’uso della funzione \texttt{sqrt} della libreria \texttt{cmath}.\\
    
    \noindent \textbf{E vietata ogni modifica alla funzione main pena l’annullamento dell’esercizio.}\\
    
    \noindent NOTA 1: La norma ad uno di un vettore è la somma dei degli elementi del vettore.
    \[||x||:=\sum_{i=1}^{n} x_i\]
    
    Ad esempio:
    \begin{itemize}
        \item Array in input: [1, 2, 4]
        \item Norma = 1 + 2 + 4 = 7
        \item Array normalizzato = [1/7, 2/7, 4/7] = [0.142857, 0.285714, 0.571428]
    \end{itemize}
    
    \noindent NOTA 2: I valori dell’array iniziale vanno inseriti utilizzando la funzione \texttt{leggi(...)} definita in \texttt{array.h} e \texttt{array.o}.\\
    La stampa dell’array normalizzato va eseguita usando la funzione \texttt{stampa(...)} definita in \texttt{array.h} e \texttt{array.o}.
    
\section*{Esercizio 4}

Scrivere un programma che calcoli il \textit{prodotto matriciale} di due matrici di numeri interi. Se le matrici non sono compatibili per il prodotto stampare un messaggio di errore.\\
\noindent La dimensione delle matrici deve essere specificata dall'utente. Il prodotto è calcolato con la funzione seguente:
    \begin{minted}
    [
        frame=lines,
        framesep=2mm,
        baselinestretch=1.2,
        fontsize=\footnotesize,
        linenos
    ]
    {C}
    int** compute_prod(int** m1, int r1, int c1, int** m2, int c2);
    \end{minted}

Dove 
\begin{itemize}
    \item \texttt{m1} è la prima matrice;
    \item \texttt{r1} è il numero di righe di m1;
    \item \texttt{c1} è il numero di colonne di m1;
    \item \texttt{m2} è la seconda matrice;
    \item \texttt{c2} è il numero di colonne di m2;
\end{itemize}
NOTA: la complessità di \texttt{compute\textunderscore prod(...)} deve essere pari o inferiore a $O(n^3)$. Ciò significa che potete implementare la funzione con l'algoritmo che si basa sulla definizione del prodotto fra matrici, oppure, se siete dei pazzi maniaci, potete implementare l'\color{blue}\href{https://it.wikipedia.org/wiki/Algoritmo_di_Strassen}{algoritmo di Strassen}\color{black}.
\end{document}