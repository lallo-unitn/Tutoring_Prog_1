\documentclass{article}
\usepackage[utf8]{inputenc}
\usepackage{hyperref}
\usepackage{float}
\usepackage{graphicx} % Gestione immagini
\usepackage{amsmath} %simboli matematici
\usepackage{minted}
\usepackage{xcolor} % to access the named colour LightGray
\definecolor{LightGray}{gray}{0.9}

\title{Esercizi 23-11-06}
\author{Riccardo Gennaro}
\date{November 2023}

\begin{document}

\maketitle

\section*{Esercizio 1}

    Scrivere un programma che gestisca l'anagrafica di un set di utenti.

    In particolare, il programma deve poter:
    \begin{itemize}
        \item aggiungere un l'anagrafica di un utente;
        \item stampare le anagrafiche ordinate per nome;
        \item stampare le anagrafiche ordinate per cognome;
        \item cercare un anagrafica per nome (corrispondenza esatta);
        \item cercare un anagrafica per cognome (corrispondenza esatta);
    \end{itemize}

    Con anagrafica si intende:

    \begin{itemize}
        \item nome;
        \item cognome;
        \item indirizzo, composto da:
        \begin{itemize}
            \item via;
            \item civico;
            \item comune;
            \item CAP;
            \item provincia;
        \end{itemize}
    \end{itemize}

    \noindent Descrivere l'anagrafica tramite \textit{struct}.    

    \noindent Il programma prevede delle dimensioni massime per gli array.

    L'utente deve poter scegliere una delle opzioni sopra. Il programma termina quando l'utente inserisce la stringa '\texttt{exit}' nel menù.
    Scrivere il programma rispettando i principi della programmazione su file multipli. 

\section*{Esercizio 2}

    Scrivere un programma che, dato un file di input
    contente delle parole, generi un secondo file in output che contenga le stesse
    parole, ma in ordine inverso rispetto al file iniziale. Per essere copiata, una
    parola deve essere composta da un numero di caratteri \textbf{pari}.\\
    
    \noindent Il programma dovrà accettare due parametri da riga di comando: il nome
    del file in input e il nome del file su cui effettuare l'output, in questo ordine.\\
    
    \noindent Il programma dovrà anche implementare dei controlli sul numero di argomenti
    passati da riga di comando e sull'apertura dei file (in caso di un file
    di input non esistente).\\
    
    \noindent Procediamo ora con un esempio. Dato in input il file input\textunderscore A, contenente le seguenti parole:\\
    
    \noindent \texttt{Impara a risolvere tutti i problemi che sono stati risolti.
    Richard Feynman}\\

    \noindent se l'eseguibile è a.out, allora il comando \texttt{./a.out input\textunderscore A output} genererà un file chiamato \texttt{output} che conterrà le seguenti parole:\\
    
    \noindent \texttt{risolti. sono problemi Impara}\\
    
    \noindent \textbf{Note}
    \begin{itemize}
        \item Per semplicità, considerate come una \textit{parola} qualsiasi stringa di caratteri
        compresa tra due spazi bianchi (e/o separatori di tabulazione,
        nuova linea e fine file). Quindi, sono da considerarsi parole anche
        stringhe come \texttt{"andato:"} o \texttt{"!esempio"}.
        \item Il file in input può contenere al massimo 10000 parole e ognuna di
        queste parole è formata da al massimo 100 caratteri.
        \item Non è consentito l'utilizzo di librerie diverse da \texttt{<iostream>} ed \texttt{<fstream>} pena \textbf{l'annullamento dell'esercizio}.
        \item E' consentito definire ed implementare funzioni ausiliarie che possano\\
        aiutarvi nella soluzione del problema.
    \end{itemize}

\newpage
    
\section*{Esercizio 3}

Scrivere un programma che calcoli il \textit{prodotto matriciale} di due matrici di numeri interi. Se le matrici non sono compatibili per il prodotto stampare un messaggio di errore.\\
\noindent La dimensione delle matrici deve essere specificata dall'utente. Il prodotto è calcolato con la funzione seguente:
    \begin{minted}
    [
        frame=lines,
        framesep=2mm,
        baselinestretch=1.2,
        fontsize=\footnotesize,
        linenos
    ]
    {C}
    int** compute_prod(int** m1, int r1, int c1, int** m2, int c2);
    \end{minted}

Dove 
\begin{itemize}
    \item \texttt{m1} è la prima matrice;
    \item \texttt{r1} è il numero di righe di m1;
    \item \texttt{c1} è il numero di colonne di m1;
    \item \texttt{m2} è la seconda matrice;
    \item \texttt{c2} è il numero di colonne di m2;
\end{itemize}
NOTA: la complessità di \texttt{compute\textunderscore prod(...)} deve essere pari o inferiore a $O(n^3)$. Ciò significa che potete implementare la funzione con l'algoritmo che si basa sulla definizione del prodotto fra matrici, oppure, se siete dei pazzi maniaci, potete implementare l'\color{blue}\href{https://it.wikipedia.org/wiki/Algoritmo_di_Strassen}{algoritmo di Strassen}\color{black}.
\end{document}