\documentclass{article}
\usepackage[utf8]{inputenc}
\usepackage{hyperref}
\usepackage{float}
\usepackage{graphicx} % Gestione immagini
\usepackage{amsmath} %simboli matematici
\usepackage{minted}
\usepackage{xcolor} % to access the named colour LightGray
\definecolor{LightGray}{gray}{0.9}

\title{Esercizi 22-11-11}
\author{Riccardo Gennaro & Jan Tomassi}
\date{November 2022}

\begin{document}

\maketitle

\section*{NOTA}

Tutti gli esercizi vanno svolti rispettando le buone pratiche di allocazione dinamica. Potete implementare su più file per esercitarvi.

\section*{Esercizio 1}

    Scrivi una funzione che prenda una stringa e restituisca un conteggio di ogni lettera nella stringa. Ad esempio, "my dog
    ate my homework" contiene 3 m, 3 o, 2 e, 2 y e uno ciascuno tra d, g, a, t, h, w, r, k.

    La funzione dovrebbe accettare un singolo argomento stringa e restituire un array allocato dinamicamente di 26 numeri interi
    che rappresenta il conteggio di ciascuna delle lettere dalla a alla z rispettivamente. La funzione dovrebbe essere case insensitive, cioè,
    contare 'A' e 'a' come l'occorrenza della lettera a. (Suggerimento: usa le funzioni di conversione da lettera a numero intero.) Non contare i caratteri non alfabetici (ad es. spazi, punteggiatura, cifre, ecc.)

    Il programma richiederà una stringa dall'utente usando getline(). Chiamando la funzione, questa stampa la frequenza di ogni lettera nella stringa. Non elencare lettere che non ricorrono almeno una volta.

    Example:
    \begin{verbatim}
        Enter a string: my dog at my homework
        Letter frequency
            a 1
            d 1
            e 1
            g 1
            h 1
            k 1
            m 3
            o 3
            r 1
            t 1
            w 1
            y 2
    \end{verbatim}

    \section*{Esercizio 2}
    Creare una matrice dinamica di tipo int[][] con dimensioni a piacere (aka allocata dinamicamente), e assegnare ad ogni cella un valore che si puo ottenere dalla funzione rand() la quale va prima inizializzata con srand(), per poi stampare tutta la matrice.

\end{document}